\documentclass{beamer}

\mode<presentation> {
	
	%\usetheme{CambridgeUS}
	\usetheme{Madrid}
	%\usetheme{Pittsburgh}
	%\usetheme{Singapore}
}

\usepackage{booktabs} 
\usepackage[utf8]{inputenc}
\usepackage[T1]{fontenc}
\usepackage[english]{babel}
\usepackage{bbding}
\usepackage{bbm}
\usepackage{tikz}
\usepackage{graphicx}
\usepackage{caption}
\usepackage{subcaption}
\usepackage{xcolor}
\usepackage{framed}
\usepackage{amsthm}
\usepackage{appendixnumberbeamer}
\usepackage{natbib}
\usepackage{multirow,array}
\usepackage{mathtools}
\usetikzlibrary{calc}

\DeclareMathOperator*{\argmax}{arg\,max}

\setbeamertemplate{theorems}[numbered]
\newtheorem{assumption}{Assumption}
\newtheorem{proposition}{Proposition}

\setbeamertemplate{navigation symbols}{}
\setbeamertemplate{footline}{}

\begin{document}
	
\title[]{"Whatever it takes": a good communication strategy?}
	
\author{Federico Innocenti\thanks{Department of Economics, University of Verona.} \and Tsung-Hsien Li\thanks{Institute of Economics, Academia Sinica}}
\date{XVIII GRASS Workshop \\
September 10, 2024} 

\begin{frame}
	\titlepage 
\end{frame}

\begin{frame}[allowframebreaks]
\frametitle{Motivation}
Central banks strategically provide information to economic agents to influence their behavior. 
\begin{itemize}
    \item In an inflation targeting regime the way market participants form their expected inflation determines the effectiveness of monetary policy.
    \item Communication helps central banks with expectation management \citep{Casiraghi2022}.
\end{itemize}
\vskip10pt
A central bank can provide information to economic agents in various forms, from technical reports to policy announcements.
\begin{enumerate}
    \item Technical reports are more precise and, thus, in principle more suitable to steer behavior in favor of its policy objective.
    \item However, it is also costly for market participants to process information, and the cost increases in its complexity \citep{Sims2003}.
    \item Technical reports can be less effective than simple speeches because the former are too complex.
\end{enumerate} 
\framebreak
The most famous example of simple but effective communication is perhaps a quote from the former ECB President Mario Draghi in 2012 during the Eurozone crisis:
\vskip5pt
\begin{quote}
Within our mandate, the ECB is ready to do \textbf{whatever it takes} to preserve the euro. And believe me, it will be enough.
\end{quote}
\vskip5pt
Why was this quote so powerful?
\begin{enumerate}
    \item It is vague. It does not provide information about how the crisis was expected to evolve or the instruments to solve it. 
    \item Yet it was enough to reassure the financial markets.
    \item \textbf{Key insight}: optimal communication is persuasive but simple.
\end{enumerate}
\framebreak
Central banks face a trade-off between policy capacity and ``popularity'', i.e., the ability to gather attention. 
\begin{itemize}
    \item The central bank can use information to alter beliefs and, thus, behavior.
    \item However, this attempt is ineffective if economic agents do not have the attention o the ability to understand the content of information.
\end{itemize}
\vskip10pt
\textbf{Research questions:}
\begin{enumerate}
    \item What is the best communication strategy for a central bank?
    \item How does it depend on the policy objective, macroeconomic conditions and agents' beliefs?
\end{enumerate}
\end{frame}

\begin{frame}
\frametitle{This paper}
We study a Bayesian persuasion model with rationally inattentive receivers (e.g., households, investors) to answer these questions. 
\vskip10pt 
Preliminary work!
\vskip10pt
We find that the central bank provides some information only if their beliefs are different from those of households. 
\begin{enumerate}
    \item The central bank can intervene to correct excessive optimism or pessimism.
    \item It intervenes when the difference in beliefs is significantly large.
    \item The response is asymmetric when economic shocks are asymmetric.
\end{enumerate}

\end{frame}

\begin{frame}
\frametitle{Relevant Literature}
\textbf{Bayesian Persuasion}:
\begin{itemize}
    \item Pioneered by \cite{aumann1995repeated} and \cite{KG2011}. 
    \item Receivers with limited attention: \cite{Bloedel2020,Lipnowski2020,lipnowski2022,Wei2021,Matyskova2021,innocenti2022can}.
\end{itemize}
\vskip10pt
\textbf{Central Bank Communication}:
\begin{itemize}
    \item A few papers use Bayesian persuasion to study the optimal central bank communication. The closest is \cite{Ko2022}.
    \item \cite{Ko2022} assumes that the central bank produces information conditional on a private forecast of the economy. Otherwise, the central bank would always perfectly reveal the state. 
    \item We do not require such assumption. The central bank can fully reveal the state but finds it optimal to provide imperfect information because households do not process complex information.
\end{itemize}
\end{frame}

\begin{frame}[allowframebreaks]
\frametitle{Model}
Partial equilibrium forward guidance model wherein we embody a Sender-Receiver game. 
\vskip10pt
We assume that the sender has commitment power: Bayesian persuasion. 
\vskip10pt
Two states of the economy: weak (i.e., $\omega_1$) or strong (i.e., $\omega_2$). $\Omega \coloneqq \{\omega_1, \omega_2\}$ is the set of states.
\vskip10pt
Two types of agents: a unit mass of households (hereafter HHs) and the central bank (hereafter CB). 
\vskip10pt 
\framebreak
CB controls inflation:
\vskip10pt
\begin{itemize}
    \item CB sets inflation $x$ as a function of the state of the economy: 
    \begin{equation*}
        x(\omega,v)\coloneqq\left\{
        \begin{array}{cc}
        x^T+v_1&  \mbox{If } \omega=\omega_1\\
        x^T-v_2   &  \mbox{If } \omega=\omega_2
        \end{array}
        \right.
    \end{equation*}
    CB may allow higher inflation than the target when the economy is weak (to help recovery) and lower when the economy is strong (to counter future inflation).
    \vskip10pt
    \item CB chooses the flexibility $v\coloneqq(v_1,v_2)$ to allow - conditional on the state of the economy $\omega\in \Omega$ - relative to its target $x^T$ for inflation.
\end{itemize} We define CB's objective function $u$ as follows:
\begin{equation}
    \label{uc}
    u \coloneqq -\left[\omega+\gamma(x^e-x)\right]^2-\alpha(x-x^T)^2
\end{equation}
where $\gamma$ is the effect of inflation surprise on unemployment, $\alpha$ is the relative importance of the inflation gap, $x^e$ is the expected inflation.
\vskip10pt
HHs form $x^e$ using the information provided by CB:
\begin{enumerate}
    \item $x^e(\mu)$ is a function of HHs beliefs $\mu$.
    \item $\mu$ can be manipulated by CB in the Sender-Receiver game: CB is the sender, whereas HHs are the receivers.
\end{enumerate}
\vskip10pt
\framebreak
Sender-Receiver game: 
\begin{enumerate}
    \item CB strategically provides information $\pi$ to HHs to influence their behavior.
    \item HHs are either naive or sophisticated. Naive HHs have limited attention and decides whether to process the CB's information. The share of naive HHs is $\delta$.
\end{enumerate}
HHs share the same prior belief $\mu_0 \in \Delta(\Omega)$, which may differ from CB's prior belief $\mu_0^c \in \Delta(\Omega)$.\\
\framebreak
The CB commits to an information structure $\pi: \Omega \to \Delta(S)$, consisting of:
\begin{enumerate}
    \item a set of messages $S=\{s_1,s_2\}$.
    \item a family of distributions $\{\pi(\cdot|\omega)\}_{\omega\in\Omega}$ over $S$.
\end{enumerate}
Each message $s$ leads to a posterior belief $\mu_s$:
\begin{align*}
    \mu_s(\omega) = \frac{\pi(s|\omega)\mu_0(\omega)}{\pi(s|\omega_1)\mu_0(\omega_1)+\pi(s|\omega_2)\mu_0(\omega_2)},
\end{align*}
As a result, $\pi$ induces a distribution of posteriors denoted by $\tau \in \Delta(\Delta(\Omega))$:
\begin{align*}
    \tau(\mu_s) = \pi(s|\omega_1)\mu_0(\omega_1)+\pi(s|\omega_2)\mu_0(\omega_2),
\end{align*}
The martingale property or Bayes plausibility condition hold: $\mathbb{E}_{\tau}(\mu_s)=\mu_0$.
\vskip5pt 
\framebreak
Each naive HH $i$ has an attention budget $c_i$, distributed according to $F(\cdot)$. 
\vskip5pt HH $i$ devotes attention to the information $\pi$ provided by the CB if and only if $c(\pi)<c_i$, where $c(\pi)$ is the cost of processing information. In particular,
\begin{align*}
    c(\pi; \chi, \mu_0) = \chi\left[H(\mu_0)-\sum_{s \in S}\tau(\mu_s) H(\mu_s)\right],
\end{align*}
where $H(\cdot)$ is the Shannon entropy defined as:
\begin{align*}
    H(\mu) & = -\sum_{\omega\in\Omega}\mu(\omega)\ln(\mu(\omega)),
\end{align*}
and $\chi>0$ is a parameter. 
\vskip5pt 
\framebreak
It follows that the mass of HHs paying attention to CB is $1-\delta F(c(\pi))$. 
\vskip5pt 
The posterior beliefs of HHs not paying attention remain as the common prior belief $\mu_0$. 
\vskip5pt 
Therefore, the expected utility of the CB is:
\begin{equation}
\label{expected_uc}
U(\pi,v)=\mathbb{E}_{\mu_0^c}[u(\mu_0,v)]\delta F(c(\pi)) + \mathbb{E}_{\{\pi,\mu_0^c\}}[u(\mu_s,v)][1-\delta F(c(\pi))]
\end{equation}
The timing is as follows:
\begin{enumerate}
    \item CB chooses $v$.
    \item CB chooses information $\pi$.
    \item Each receiver devoting attention receives a message and updates beliefs.
\end{enumerate}
\vskip5pt
\framebreak
We solve the model by backward induction.
\begin{itemize}
    \item We start from the information design, taking as given the flexibility $v$. The second component in \eqref{uc} does not depend on the expected inflation $x^e$. Therefore, it is irrelevant for CB when designing $\pi$.
\end{itemize}
\vskip5pt
We make the following assumptions:
\begin{assumption}
    \label{Ass1}
    HHs form a rational inflation expectation:
    \begin{equation}
    \label{expectedinflation}
    x^e(\mu)=\mu x(\omega_1)+(1-\mu)x(\omega_2)=x^T+v_1\mu-v_2(1-\mu)
    \end{equation}
\end{assumption}
\begin{assumption}
\label{Ass2}
    Attention is uniformly distributed: $F(\cdot)=U[0,1]$ and $\chi=\frac{1}{\ln(2)}$.
\end{assumption}
\end{frame}


\begin{frame}[allowframebreaks]
    \frametitle{Results}
    Our framework encompasses several dimensions:
    \begin{enumerate}
        \item Heterogeneous beliefs
        \item Asymmetric shocks
        \item Inattentive households
    \end{enumerate}
    \vskip10pt
    We present preliminary results:
    \begin{enumerate}
        \item Analytical analysis for a symmetric benchmark.
        \item Numerical results for the full model.
    \end{enumerate}
    \vskip10pt
    \framebreak
    Under Assumptions \ref{Ass1}-\ref{Ass2}, CB's problem is:
    \begin{equation}
    \begin{split}
        \min_{\pi} \ & \ \delta c(\pi)\Bigg\{\mu_0^c\left[\omega_1-\gamma(v_1+v_2)(1-\mu_0)\right]^2+\\
        \ & \ +(1-\mu_0^c)\left[\omega_2+\gamma(v_1+v_2)\mu_0\right]^2\Bigg\}+\\
        \ & \ +[1-\delta c(\pi)]\Bigg\{\mu_0^c\bigg[\pi(s_1|\omega_1)(\omega_1-\gamma (v_1+v_2) (1-\mu_1))^2+\\
        \ & \ +\pi(s_2|\omega_1)(\omega_1-\gamma (v_1+v_2) (1-\mu_2))^2\bigg]+ \\
        \ & \ +(1-\mu_0^c)\bigg[\pi(s_2|\omega_2)(\omega_2+\gamma (v_1+v_2) \mu_2)^2+\\
        \ & \ +\pi(s_1|\omega_2)(\omega_2+\gamma (v_1+v_2) \mu_1)^2 \bigg]\Bigg\}
    \end{split}
    \end{equation}
    We need two further assumptions in the benchmark:
    \begin{assumption}
        \label{Ass3}
        The unemployment shocks are symmetric, that is, $\omega_1=-\omega_2=\omega$.
    \end{assumption}
    \begin{assumption}
        \label{Ass4}
        Prior beliefs are homogeneous and neutral, that is, $\mu_0=\mu_0^c=\frac{1}{2}$.
    \end{assumption}
    Under assumptions \ref{Ass3}-\ref{Ass4}, the F.O.C.s are symmetric:
    \begin{enumerate}
        \item We focus on symmetric solutions such that $\pi(s_1|\omega_1)=\pi(s_2|\omega_2)=x$;
        \item $x\in\left[\frac{1}{2},1\right]$ represents the precision of CB's recommendations.
    \end{enumerate}
    \begin{proposition}[Stage two with $\delta=0$]
    \label{Prop1}
    Under Assumptions \ref{Ass1}, \ref{Ass3}, and \ref{Ass4}, the CB's information design problem has a corner solution: $\pi$ is either fully informative (i.e., $x=1$) or uninformative (i.e., $x=\frac{1}{2}$). 
\end{proposition} 
Indeed, the F.O.C.s reduces to the following condition:
\begin{equation}
    \gamma(v_1+v_2)[4\omega-\gamma(v_1+v_2)](2x-1)\gtreqqless 0
    \end{equation}
For any $x$, the former is always negative (positive) if $\gamma(v_1+v_2)[4\omega-\gamma(v_1+v_2)]<0$ ($>0$). 
\vskip10pt Therefore, $x=1$ is the global min if $\gamma(v_1+v_2)[4\omega-\gamma(v_1+v_2)]<0$. Otherwise, $x=\frac{1}{2}$ is the global min.
\vskip5pt 
\framebreak
Given Proposition \ref{Prop1}, and the fact that $\gamma,\omega,v_1$ and $v_2$ are all positive, $\pi$ is fully informative if and only if $4\omega<\gamma(v_1+v_2)$. The optimal solution to the first stage is described in the following proposition.

\begin{proposition}[First stage with $\delta=0$]
    \label{Prop2}
    Under Assumptions \ref{Ass1}, \ref{Ass3}, and \ref{Ass4}, the optimal flexibility provided by the CB is
    \begin{equation}
        v_1=v_2=\left(\frac{\gamma}{\gamma^2+\alpha}\right)\omega
    \end{equation}
    which implies that the optimal information design is uninformative.
\end{proposition}
    \begin{proposition}[$\delta\neq0$]
    \label{Prop2}
        Under Assumptions \ref{Ass1}-\ref{Ass4}, the solution to the CB's information design problem solves the following condition:
        \begin{small}
        \begin{equation}
            \gamma(v_1+v_2)[4\omega-\gamma(v_1+v_2)]\left\{[1-\delta c(\pi)](2x-1)+2\delta c^\prime(\pi)\left[x(1-x)-\frac{1}{4}\right]\right\}=0
        \end{equation}
        \end{small}
        where
        \begin{eqnarray}
            c(\pi)=\chi \left[\ln(2)+x\ln(x)+(1-x)\ln(1-x)\right] \\
            c^\prime(\pi)=\frac{\chi}{2}\left[\ln\left(x\right)-\ln(1-x)\right]
        \end{eqnarray}
        Provided $4\omega\geq\gamma(v_1+v_2)$, the problem has corner solutions: either $x=\frac{1}{2}$ or $x=1$. The optimal flexibility is the same as before. Thus, in a symmetric setting, CB provides no information.
    \end{proposition}
    We use numerical analysis to solve the CB's problem when relaxing assumptions \ref{Ass3}-\ref{Ass4}.
    \vskip10pt
    We focus on the information design stage, taking as given $v_1,v_2$. 
    \vskip10pt We focus on the case where $4\omega\geq\gamma(v_1+v_2)$
    \begin{table}[htp!]
    \centering
    \begin{tabular}{@{}ccccccccc@{}}
    \toprule
    $\delta$ & $\omega_1$ & $\omega_2$ & $\mu_0$ & $\mu_0^c$ & $\gamma$ & $x_T$ & $v_1$ & $v_2$ \\ \midrule
    0.5      & 1          & -1         & 1/2     & 1/2       & 1      & 2     & 1 & 1    \\ \bottomrule
    \end{tabular}
    \caption{Benchmark Parameters}
    \label{tab:bchmrk_param}
    \end{table}
    \framebreak
    \begin{figure}[htp!]
    \centering
    \includegraphics[width=0.49\textwidth]{figures/V8/delta_1/fig_optimal_π_across_μ_0.pdf}
    \includegraphics[width=0.49\textwidth]{figures/V8/delta_1/fig_posterior_across_μ_0.pdf}
    \end{figure}
    CB intervenes to fix HHs misperceptions: excessive optimism or pessimism about the state of the economy
    \framebreak
    \begin{figure}[htp!]
    \centering
    \includegraphics[width=0.49\textwidth]{figures/V8/delta_1/fig_optimal_π_across_μ_0_c_μ_0_0.1.pdf}
    \includegraphics[width=0.49\textwidth]{figures/V8/delta_1/fig_posterior_across_μ_0_c_μ_0_0.1.pdf}
    \end{figure}
    In the graph, $\mu_0 = 0.1$. CB intervenes when there is a sufficiently large difference in prior beliefs.
    \vskip10pt
    \framebreak
    What if the bad shocks become more devastating? 
    \vskip10pt
    The bad state is often associated with much worse outcomes than the good state. 
    \vskip10pt
    In our setting, the unemployment rate soars in the bad state but drops mildly in the good state. That is, $\omega_1 > -\omega_2$.
    \vskip10pt Thus, we consider two cases:
    \begin{enumerate}
        \item $\omega_1 = \frac{3}{2}$;
        \item $\omega_1 = 2$
    \end{enumerate}
    while holding $\omega_2 = -1$. 
\vskip10pt
\framebreak
When $\omega_1 = \frac{3}{2}$, the results are:
\begin{figure}[htp!]
\centering
\includegraphics[width=0.49\textwidth]{figures/V8/delta_1/fig_optimal_π_across_μ_0_ω_1_1.5_ω_2_-1.pdf}
\includegraphics[width=0.49\textwidth]{figures/V8/delta_1/fig_posterior_across_μ_0_ω_1_1.5_ω_2_-1.pdf}
\end{figure}
\vskip10pt
\framebreak
When $\omega_1 = 2$, the results are:
\begin{figure}[htp!]
\centering
\includegraphics[width=0.49\textwidth]{figures/V8/delta_1/fig_optimal_π_across_μ_0_ω_1_2_ω_2_-1.pdf}
\includegraphics[width=0.49\textwidth]{figures/V8/delta_1/fig_posterior_across_μ_0_ω_1_2_ω_2_-1.pdf}
\end{figure}
The asymmetric information provision allows CB to increase the inflation surprise when more needed, that is, when the economy is weak.
\end{frame}


\begin{frame}
	\frametitle{Conclusion}
    We address the problem of a central bank that wishes to guide the economy but faces inattentive households.
    \vskip10pt
    We are still working on several dimensions:
    \begin{enumerate}
        \item The optimal flexibility $v$.
        \item Calibrate the model to different monetary areas (e.g. Euro zone) to characterize the optimal information policy of each central bank.
        \item Imperfect expectations by households: better definition for naivety.
    \end{enumerate}
    \vskip10pt
    We are happy to hear your suggestions!
\end{frame}

\begin{frame}
	\begin{huge}
		\centerline{Thank you for your \textbf{limited} attention!}
	\end{huge}
\end{frame}

\begin{frame}<presentation:0>[noframenumbering]
	\bibliographystyle{ecta}
	\bibliography{references}
\end{frame}


\end{document}